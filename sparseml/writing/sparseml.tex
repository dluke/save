\documentclass{article}

\usepackage[utf8]{inputenc}
\usepackage{amsfonts}
\usepackage{amstext}
\usepackage{amssymb}
\usepackage{bm}
\usepackage{xcolor}   
\usepackage[hidelinks]{hyperref}
\usepackage{amsmath}

\usepackage{booktabs,caption}
\usepackage{threeparttable}
\usepackage{adjustbox}
\usepackage{parskip}
\usepackage{color, colortbl}

\usepackage{todonotes}


% \usepackage[style=authoryear-comp,citestyle=numeric,maxcitenames=1,backend=biber]{biblatex}
\usepackage[style=numeric,citestyle=numeric,backend=biber,sorting=none]{biblatex}
\addbibresource[]{sparse.bib}


\graphicspath{ {./svg/} }

\newcommand{\citetemp}[1]{(#1)}
\newcommand{\reftemp}[1]{(#1)}

%math
\newcommand{\vct}[1]{\bm{#1}} % vector symbol formatting 
\newcommand{\mtx}[1]{\bm{#1}} % matrix symbol formatting
\newcommand{\vnorm}[1]{||#1||}
\newcommand{\vunit}[1]{\hat{#1}}
\newcommand{\mskew}[1]{\tilde{#1}}
\newcommand{\pderiv}[2]{\frac{\partial#1}{\partial#2}}
\newcommand{\rpm}{\raisebox{.2ex}{$\scriptstyle\pm$}}
\newcommand{\mean}[1]{\langle{#1}\rangle} 

\newcommand{\citetodo}{[]}
\newcommand{\reflater}{{\color{red} \{ref\}}}

%my operators
\DeclareMathOperator{\fskew}{skew}
\DeclareMathOperator{\Project}{Project}

\title{Sparse representations of single cell dynamics}

\author{Daniel Barton}

\begin{document}

\maketitle


\section{Introduction}

We want to obtain optimal reduced dimension representations of biological single
cell tracking data. Such representations can for example be used to compare 
trajectories of different organisms and identify and 
quantify their common abstract features.

Motile microorganisms commonly have a preferred direction of motion on intermediate 
timescales while having highly stochastic behaviour on short timescales.
This short timescale stochasiticity is the expected result of, for example
the quasi random molecular dyanmics of actin filaments and myosin motors
or the stochasitc growth, surface attachent and retraction of bacteria type IV pili (TFP).

We  use ideas from the automated machine learning of the governing equations 
of physical systems from trajectory data, which works by obtaining 
sparse representations of a library of basis functions. This framework is
extremely powerful and general but relies on the construction of a suitable
library of basis functions in which the dynamics under examination are
sparesly represented.

We start with the following auto regressive model of a biological unit with 
orientation $\vct{b} = b(\cos{\theta}, \sin{\theta})$ and velocity $\vct{v}$,

\begin{equation}
\begin{split}
    v_{i+1}^\parallel & = q^\parallel v_i^\parallel + \ldots +  a^\parallel \vct{\hat{n}}_{i+1} \\
    v_{i+1}^\bot & = q^\bot v_i^\bot + \ldots + a^\bot \vct{\hat{n}}_{i+1} \\ 
    \theta_{i+1}  & = \theta_i + a_r \vct{\hat{n}}_{i+1}
\end{split}
\end{equation}

where $v_i^\parallel = (\vct{v}_i \cdot \vct{b}_i)$, 
$v_i^\bot = \vct{v}_i - v_i^\parallel\vct{\hat{b}}_i$ .



\section{Construct test data}

\section{Multiscale Entropy Analysis}


%-----------------------------------------------------------------------

\medskip

\printbibliography

% =================================================================
\end{document}
% ------------------------------------------------------------------------
